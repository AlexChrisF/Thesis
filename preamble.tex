\documentclass[
BCOR=5mm,           % Binderkorrektur von 5mm vorsehen
DIV10,              % Seite in X Kästchen einteilen (Siehe Koma-Script Guide)
%DIVcalc,           % Besten DIV Wert berechnen (Siehe Koma-Script Guide)
fontsize=11pt,      % Schriftgröße 11 Punkte
oneside,            % Einseitig
parskip,            % Paragraphen nicht einrücken
headsepline,        % Kopfzeile nach unten durch Linie abgrenzen (scrheadings)
%footbotline,       % Fußzeile nach unten durch Linie abgrenzen (scrheadings)
plainheadsepline,   % Kopfzeile nach unten durch Linie abgrenzen (scrplain)
plainfootbotline,   % Fußzeile nach unten durch Linie abgrenzen (scrplain)
%headtopline,       % Kopfzeile nach oben durch Linie abgrenzen (scrheadings)
footsepline,        % Fußzeile nach oben durch Linie abgrenzen (scrheadings)
plainheadtopline,   % Kopfzeile nach oben durch Linie abgrenzen (scrplain)
plainfootsepline,   % Fußzeile nach oben durch Linie abgrenzen (scrplain)
headexclude,        % Kopfzeile nicht als Teil des Inhalts setzen
footexclude,        % Fußzeile nicht als Teil des Inhalts setzen
%bibtotocnumbered,  % Literaturverzeichnis nummeriert ins Inhaltsverzeichnis aufnehmen
bibtotoc,           % Literaturverzeichnis ins Inhaltsverzeichnis aufnehmen
%liststotocnumbered,% Sonstige Verzeichnise nummeriert ins Inhaltsverzeichnis aufnehmen
liststotoc,         % Sonstige Verzeichnise ins Inhaltsverzeichnis aufnehmen
idxtotocnumbered    % Index nummeriert ins Inhaltsverzeichnis aufnehmen
%idxtotoc           % Index ins Inhaltsverzeichnis aufnehmen
]{scrbook}          % Koma-Script Klasse zum setzen eines Buchs

% Die "Standard-Header" für deutsche Dokumente
\usepackage[latin1]{inputenc}    % ISO-8859-1 bzw. Latin1 als Encoding
\usepackage[T1]{fontenc}         % T1 Schriften verwenden (sieht besser aus)
\usepackage[english]{babel}      % Neue deutsche Rechtschreibung und Übersetzungen

% "Schönere" Schriften einbinden
\usepackage{mathpazo}            % Serifen-Font mit passendem Math-Font
\usepackage[scaled=.95]{helvet}  % Serifenloser Font passend zu mathpazo
\usepackage{courier}             % "Schönerer" Festbreiten-Font

% Koma-Script Paket zum setzen vom Kopf- und Fußzeilen einbinden
\usepackage{scrpage2}
% Seitenstil für normale Seiten auf scrheadings setzen
% Für Kapitelanfang und ähnliches wird scrplain verwendet
\pagestyle{scrheadings}
% Kopf- und Fußzeile löschen
\clearscrheadfoot
% Automarkierungen verwenden \automark[rechts]{links}
% Statt \leftmark und \rightmark kann dann bei
% Koma-Script einfach \headmark verwendet werden
\automark[section]{chapter}
% Kopfzeile für scrplain und scrheadings setzen
% \*head[scrplain]{scrheadings}
%\ihead[Innen]{Innen}
%\chead[Mitte]{Mitte}
\ohead[\sffamily\scshape\bfseries\large\headmark]
{\sffamily\scshape\bfseries\large\headmark}
% Fußzeile für scrplain und scrheadings setzen
% \*foot[scrplain]{scrheadings}
%\ifoot[Innen]{Innen}
%\cfoot[Mitte]{Mitte}
\ofoot[\sffamily\thepage]{\sffamily\thepage}
% Trennlinien für Kopf- und Fußzeile formatieren
% Siehe Optionen der Dokumentklasse
%\setheadtopline{2pt}
\setheadsepline{.4pt}
\setfootsepline{.4pt}
%\setfootbotline{2pt}

% Paket zum Einbinden von Quellcode als Listings
% Hinweis: UTF-8 Encoding führt zu Problemen mit Umlauten
\usepackage{listings}
% Formatierung der Listings
\lstset{
captionpos=b,                % Beschriftung unterhalb (bottom)
numbers=left,                % Zeilennummern links
frame=trbl,                  % Rahmen zeichnen (top, right, bottom, left)
basicstyle=\small\ttfamily,  % Festbreitenschrift verwenden (small)
language=Java                % Sprache auf Java einstellen
}

% Paket für definierte Übersetzungen einbinden
\usepackage[USenglish]{translator}

% Paket für Stichwort- Abkürzungs- und sonstige Verzeichnisse einbinden
\usepackage[
nonumberlist, % Keine Seitenzahlen anzeigen
acronym,      % Abkürzungsverzeichnis erstellen
toc,          % In Inhaltsverzeichnis aufnehmen
%section       % Verzeichniseintrag als Section
]{glossaries}

% Ein eigenes Verzeichnis definieren (Smbolverzeichnis)
% Das Stichwort- und Abkürzungsverzeichnis wird analog vordefiniert
% Siehe makeindex Aufrufe - Hier werden die Dateiendungen festgelegt
\newglossary[slg]{symbolslist}{syi}{syg}{Symbolverzeichnis}

% Den Punkt am Ende der Beschreibung deaktivieren
% \renewcommand*{\glspostdescription}{}

% Stichwort-, Abkürzungs- und Symbolverzeichnis erzeugen
\makeglossaries

% Paket für Wortindex einbinden
\usepackage{makeidx}

% Wortindex erzeugen
\makeindex

% Paket zum generieren von Blindtext
\usepackage{blindtext}

% Paket zum Einbinden von Bildern
\usepackage{graphicx}

%              
% WORKAROUND, damit lstlistoflistings funktioniert:
% Quelle: http://www.komascript.de/node/477
%
\makeatletter
\@ifundefined{float@listhead}{}{%
    \renewcommand*{\lstlistoflistings}{%
        \begingroup
    	    \if@twocolumn
                \@restonecoltrue\onecolumn
            \else
                \@restonecolfalse
            \fi
            \float@listhead{\lstlistlistingname}%
            \setlength{\parskip}{\z@}%
            \setlength{\parindent}{\z@}%
            \setlength{\parfillskip}{\z@ \@plus 1fil}%
            \@starttoc{lol}%
            \if@restonecol\twocolumn\fi
        \endgroup
    }%
}
\makeatother

% My stuff

\usepackage[onehalfspacing]{setspace}
\usepackage{svg}
\usepackage{amsmath}
\usepackage{float}
\usepackage{listings}


\lstdefinelanguage{JavaScript}{
	morekeywords={typeof, new, true, false, catch, function, return, null, catch, switch, var, if, in, while, do, else, case, break},
	morecomment=[s]{/*}{*/},
	morecomment=[l]//,
	morestring=[b]",
	morestring=[b]'
}


\lstset{%
	% Basic design
	frame=l,
	% Line numbers
	xleftmargin={0.75cm},
	numbers=left,
	stepnumber=1,
	firstnumber=1,
	numberfirstline=true,
	% Code design   
	% Code
	alsolanguage=JavaScript,
	alsodigit={.:;},
	tabsize=2,
	showtabs=false,
	showspaces=false,
	showstringspaces=false,
	extendedchars=true,
	breaklines=true,        
	% Support for German umlauts
	literate=%
	{�}{{\"O}}1
	{�}{{\"A}}1
	{�}{{\"U}}1
	{�}{{\ss}}1
	{�}{{\"u}}1
	{�}{{\"a}}1
	{�}{{\"o}}1
}


\usepackage{tikz}
\usetikzlibrary{shadows}
\newcommand*\keystroke[1]{%
	\tikz[baseline=(key.base)]
	\node[%
	minimum width=1.2em,
	draw,
	fill=white,
	drop shadow={shadow xshift=0.25ex,shadow yshift=-0.25ex,fill=black,opacity=0.75},
	rectangle,
	rounded corners=2pt,
	inner sep=1pt,
	line width=0.5pt,
	font=\scriptsize\sffamily
	](key) {#1\strut}
	;
}


\usepackage{forest}

\definecolor{folderbg}{RGB}{124,166,198}
\definecolor{folderborder}{RGB}{110,144,169}

\def\Size{4pt}
\tikzset{
	folder/.pic={
		\filldraw[draw=folderborder,top color=folderbg!50,bottom color=folderbg]
		(-1.05*\Size,0.2\Size+5pt) rectangle ++(.75*\Size,-0.2\Size-5pt);  
		\filldraw[draw=folderborder,top color=folderbg!50,bottom color=folderbg]
		(-1.15*\Size,-\Size) rectangle (1.15*\Size,\Size);
	}
}
