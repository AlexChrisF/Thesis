\chapter{Motivation}

These days a lot of data is generated to improve the way cars are used or phones are unlocked. Autonomous driving and face recognition are only two examples where a lot of data is needed to improve algorithms and with that the devices these algorithms are used for. 

The most interesting area large collections of data are used is health care. Modern technologies give possibilities that have not been there before. Artificial intelligence makes it possible to identify a malignant melanoma by discriminating it from a harmless mole with a normal cellphone camera.

However, there is still a lot of high potential data unused. One reason for that is computer laymen don't know how to use the data they have. It seems to be a privilege to Computer Scientists and Computer Enthusiasts to take advantage of that data and the potential it holds. 

For example biologists and doctors going through Hundreds and Thousands of samples, giving them a tool to automatize there daily work flow would be a huge ease.

Goal of this bachelor thesis is to evaluate on how to give normal users, users without knowledge of how to install a Python environment or dealing with databases, the possibility to use all advantages of machine learning. By that this projects follows the modest aim to democratize artificial intelligence.