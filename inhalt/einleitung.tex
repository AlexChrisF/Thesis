\chapter{Motivation}

These days a lot of data is collected to improve the way cars are used or phones are unlocked. Autonomous driving and face recognition are two examples where large data sets are required to improve algorithms and thereby make cars more capable and phones easier to use. 

One of the most interesting areas large collections of data can be
used with significant benefits is in health care. Modern
technologies provide
possibilities beyond anything that could have been 
conceived just a few years back. 
For example, artificial
intelligence enables identifying malignant melanomas by
discriminating them from harmless moles using a standard cellphone camera.

However, a lot of high potential data is still being unused. 
One reason is that computer laymen often don't know how to use the data they already have. 
It seems to be a privilege of computer scientists and computer enthusiasts to take advantage of such data and the potential it holds. 

For example, if biologists and doctors had a tool that would 
help them classify and structure their specimen
in an easy way, or even automate such work, it could vastly
increase their productivity and diagnostic capacity.

The objective of this bachelor thesis is to evaluate how regular
users without any computer science background could access the
potential of machine learning in their daily work to assist them
in performing intellectually demanding tasks. 

Stated a little pathetically: This project wants to help democratizing access to artificial intelligence.