\chapter{CYTO AI}

CYTO AI is the first fully web-based image labeling and classifying tool publicly available. Based on the JavaScript frameworks React and TensorFlow.js,
all advantages of modern machine learning and labeling
technology could be employed in a zero installation effort,
completely platform independent, and fully browser based
solution.

This rich client application does not require the upload of
any data to a server. This helps both, privacy concerns as well as performance. The following sections shall explain how CYTO AI works and which techniques and architectures were used to build it.

\begin{figure}[H]
	\centering
	\includegraphics[width=0.8\linewidth]{bilder/cyto/cyto.png}
	\caption{CYOTO AI}
	\label{fig:COMPONENT}
\end{figure}


\subsection{Workflow overview}

The workflow starts with the creation of categories. Each image to be classified will have to be assigned to a category.

Now the images need to be made available to the application.
They are being uploaded to the local browser cache.
It is possible to select entire folders which will result in being all images in that folder and their sub-folders being uploaded.

Once all images have been uploaded a knowledgeable user has
to classify them by dragging them to the respective category field.

After sufficient samples haves been categorized the network
can be trained. Depending the amount of images this can take more or less time. The training results in a trained network.
Which now can be used to classify the previously
unclassified images. This machine labeling needs to
be controlled by a knowledgeable user
and erroneous annotations have to be manually corrected.

\begin{figure}[H]
	\centering
\includegraphics[scale=0.6]{bilder/cyto/Ablaufdiagramm.png}
	\caption{Workflow}
	\label{fig:Workflow}
\end{figure}

Thereafter the network is trained again. This procedure is 
repeated until the labeling error rate by the trained
network is satisfactory.

Now the labeled images can be exported and used by others
to train their networks. In future it will be possible to
export the network parameters so that a complete trained
can be shared with others.


\section{Working with CYIO AI}

\subsection{Creating categories}
By clicking on the add button it is possible to create a new
category. In order to change a category, the category needs to be deleted, by clicking on the delete Icon and a new category needs to be created. Future versions will support the editing of an existing category so it will not be necessary anymore to delete a category in order to change the name.
\begin{figure}[H]
	\centering
	\includegraphics[scale=0.8]{bilder/cyto/categories.png}
	\caption{Create a ctagory}
	\label{fig:Category}
\end{figure}

\subsection{Uploading images}
In order to upload images the upload image button can be clicked. In Chrome, Firefox it is possible to select 
whole folders. The picture must be in the portable network graphics format (PNG).

\begin{figure}[H]
	\centering
	\includegraphics[scale=0.8]{bilder/cyto/UploadImages.png}
	\caption{Image upload}
	\label{fig:ImageUpload}
\end{figure}


\subsection{Labeling images}

There are several ways to annoate an image. It is possible
to drag an image and drop it on the wished category.
A different way is to click on one picture in order to
select it an then use the keyboard to annotate the image.
Following keys are supported to annotate an image:


\begin{itemize}
	\item \keystroke{ 1 } \keystroke{2} \keystroke{3} \keystroke{4} \keystroke{5} \keystroke{6} \keystroke{7}
	\keystroke{8} \keystroke{9} \keystroke{0} where \keystroke{1} is the first category in list.
	\item \keystroke{$\Leftarrow$} backslash to delete a given category
\end{itemize}

Further it is possible to navigate through all images with the arrow keys \keystroke{$\Uparrow$} to go up in row and 
\keystroke{$\Downarrow$} to go down in row. Further \keystroke{$\Rightarrow$} to go right in column and \keystroke{$\Leftarrow$} to go left in column.


\subsection{Classifying images}

For classifying the fit button needs to be clicked.
The neural network will be trained and all unlabeled
images will be categorized.

\begin{figure}[H]
	\centering
	\includegraphics[scale=0.8]{bilder/cyto/Fit.png}
	\caption{Classify images}
	\label{fig:Clssify}
\end{figure}

By that all unlabeled images will be given a category and a number will be shown under the image. This number is the probability an image belongs to the given category.

\subsection{Exporting}

To save all labels and categories and settings the button save can be pressed. To import again the "open" button can be used. The files will be exported in a JSON format.

\begin{figure}[H]
	\centering
	\includegraphics[scale=0.8]{bilder/cyto/OpenSave.png}
	\caption{Export and import labels}
	\label{fig:ExportImport}
\end{figure}

\subsection{Filtering and sorting}
 
Also it is possible to blend out certain categories by
clicking on the category, clicking another time will blend
in the category. That gives the possibility to only show
certain categories or to only show unlabeled images.
If all categories are blended out it is still possible to
annotate, newly labeled image will stay visible.
 
\begin{figure}[H]
 	\centering
 	\includegraphics[scale=0.8]{bilder/cyto/BlendedOut.png}
 	\caption{Blending out images with category}
 	\label{fig:BlendingOut}
\end{figure}
  
If pictures were labeled it makes sense to sort them, this
is possible by clicking the sort button. This improves the
overview and makes reviewing labels easier. Uncategorized
images will always appear at the top.
 
Another feature that improves the overview is the slider. The slider makes it possible to adjust the number of displayed images per row. 

\begin{figure}[H]
	\centering
	\includegraphics[scale=0.8]{bilder/cyto/Slider.png}
	\caption{Slider adjusts number of pictures displayed per row}
	\label{fig:Slider}
\end{figure}

\section{IDE setup}
Following IDE setup was chosen to develop CYTO AI.
Visual Studio Code was introduced in 2016 it porovides integrated GIT support as weel as easy debugging and syntax highlighting, intelligent code completion.

\begin{figure}[H]
	\centering
	\includegraphics[width=\linewidth]{bilder/cyto/IDE.png}
	\caption{IDE setup}
	\label{fig:IDE}
\end{figure}

\section{Package management}

\section{The edit-build-debug cycle}

\subsection{System architecture overview}
The following diagram shall give an overview of how responsibilities are distributed over the system.

\begin{forest}
	for tree={
		font=\ttfamily,
		grow'=0,
		child anchor=west,
		parent anchor=south,
		anchor=west,
		calign=first,
		inner xsep=7pt,
		edge path={
			\noexpand\path [draw, \forestoption{edge}]
			(!u.south west) +(7.5pt,0) |- (.child anchor) pic {folder} \forestoption{edge label};
		},
		% style for your file node 
		file/.style={edge path={\noexpand\path [draw, \forestoption{edge}]
				(!u.south west) +(7.5pt,0) |- (.child anchor) \forestoption{edge label};},
			inner xsep=2pt,font=\small\ttfamily
		},
		before typesetting nodes={
			if n=1
			{insert before={[,phantom]}}
			{}
		},
		fit=band,
		before computing xy={l=15pt},
	}
	[CYTO AI
	[config
	]
	[lib
	[Access
	]
	[Plugin
	]
	[file.txt,file
	]
	]
	[templates
	]
	[tests
	]
	]
\end{forest}


API contains all files and functionality needed for the machine learning classifying task. Folder src contains all components, tests and Redux related files. 
In order to explain how this structure works a further component shall be added.


\section{Adding a component}
First a new component "CategoryList" is created. CategoryList shall simply show a list of categories. 




\lstinputlisting{code/cyto/categoryList.js}

The data shall be fetched from a variable categories, placed in the Redux store. In order to fetch this data the component shall be connected to the Redux store so a High-Order Component CategoryListConnector is created. High-Order components wrap a component and add some functionalities in this case it adds the possibility to access the categories array in Redux store. 

\lstinputlisting[caption=CategoryList component]{code/cyto/connectedCategory.js}

The directory will now look like this:

\begin{forest}
	for tree={
		font=\ttfamily,
		grow'=0,
		child anchor=west,
		parent anchor=south,
		anchor=west,
		calign=first,
		inner xsep=7pt,
		edge path={
			\noexpand\path [draw, \forestoption{edge}]
			(!u.south west) +(7.5pt,0) |- (.child anchor) pic {folder} \forestoption{edge label};
		},
		% style for your file node 
		file/.style={edge path={\noexpand\path [draw, \forestoption{edge}]
				(!u.south west) +(7.5pt,0) |- (.child anchor) \forestoption{edge label};},
			inner xsep=2pt,font=\small\ttfamily
		},
		before typesetting nodes={
			if n=1
			{insert before={[,phantom]}}
			{}
		},
		fit=band,
		before computing xy={l=15pt},
	}  
	[system
	[config
	]
	[lib
	[Access
	]
	[Plugin
	]
	[file.txt,file
	]
	]
	[templates
	]
	[tests
	]
	]
\end{forest}


Further it shall be possible to add another category to the category list, therefore an action is needed that takes the new category name and the category color as a payload.

\lstinputlisting[caption=Actions for changing categories]{code/cyto/actions.js}

A reducer will now be called given and the action will be passed to it.

\lstinputlisting[caption=Reducer for changing categories]{code/cyto/categoryReducer.js}

The structure will now look like this:

\begin{forest}
	for tree={
		font=\ttfamily,
		grow'=0,
		child anchor=west,
		parent anchor=south,
		anchor=west,
		calign=first,
		inner xsep=7pt,
		edge path={
			\noexpand\path [draw, \forestoption{edge}]
			(!u.south west) +(7.5pt,0) |- (.child anchor) pic {folder} \forestoption{edge label};
		},
		% style for your file node 
		file/.style={edge path={\noexpand\path [draw, \forestoption{edge}]
				(!u.south west) +(7.5pt,0) |- (.child anchor) \forestoption{edge label};},
			inner xsep=2pt,font=\small\ttfamily
		},
		before typesetting nodes={
			if n=1
			{insert before={[,phantom]}}
			{}
		},
		fit=band,
		before computing xy={l=15pt},
	}  
	[system
	[config
	]
	[lib
	[Access
	]
	[Plugin
	]
	[file.txt,file
	]
	]
	[templates
	]
	[tests
	]
	]
\end{forest}

In short for each added component that needs access to the states a connector needs to be created. Actions and reducers also need to be created to change the state, if not already existing.



\section{The machine learning API}
The machine learning API was built with TensorFlow.js. Every time the "fit" button was clicked an asynchronous API call is made. The calls look like following:

\lstinputlisting[caption=Machine learning API ]{code/cyto/classifier.js}

Where dataset is an instance for handling the data. And imageTags are the HTML . The API is using a CNN Network provided over a Content Delivery Network (CDN).

\section{Performance}








