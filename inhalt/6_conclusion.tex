\chapter{Summary and Outlook}

As Samuel Buttler used to say: "Brevity is very good where,
where we are and are not understood." So this summary shall be rather brief.

At the start of the project we had a very powerful but
somewhat difficult to install and use classification
 application written in Python with more than 27.000 lines of code. 
It looked like
an almost impossible task to port such significant piece of
work to an entire new architecture and completely new technologies in just a few months.

Now, at the end of this work and in cooperation with some 
highly talented people, it looks like the decision to
 move to the new web centric development paradigm has 
 paid for itself.

All major technological issues could be resolved and there seems to be good reason to believe that continuing this way
would result in less development effort and better user
experience compared to the traditional way of
implementation.

What is still to be done?

Thus far, the tool can classify entire images. A big step forward would be the capability to segment images, 
that is to locate image areas with objects of interest
prior to the classification process. 

However, this would place considerable demands on the tool and would
require another substantial development effort. The user interface would have to be redesigned so that 
images could be annotated.  

Furthermore, it will soon be possible to export models (trained nets) with Tensorflow.js. The declared aim of Goggle is to make this possible in the upcoming versions (current version 0.10). It is conceivable to make the exported models accessible to other users via a web platform. 
